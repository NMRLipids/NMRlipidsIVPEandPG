\documentclass{article}

%\usepackage{longtable}

\title{Tables of PDB structures used to analyze protein-bound lipids}

\begin{document}



\begin{table}[]
    \centering
    \begin{tabular}{c|c}
PDBID & molecule name \\
\hline
1ein & Lipase \\ 
3uu5 & Proton-gated ion channel \\ 
3p4w & Proton-gated ion channel \\ 
3p50 & Proton-gated ion channel \\ 
3uu6 & Proton-gated ion channel \\ 
3uu8 & Proton-gated ion channel \\ 
3uub & Proton-gated ion channel \\ 
4dos & Nuclear receptor subfamily 5 group A member 2 \\ 
3h1j & MITOCHONDRIAL UBIQUINOL-CYTOCHROME-C \\
     &REDUCTASE COMPLEX CORE PROTEIN I \\ 
1lpa & Colipase \\ 
4hfc & Proton-gated ion channel \\ 
5mvm & Proton-gated ion channel \\ 
5ejz & Cellulose synthase catalytic subunit [UDP-forming] \\ 
5j0z & Proton-gated ion channel \\ 
5mzr & Proton-gated ion channel \\ 
5mvn & Proton-gated ion channel \\ 
5mur & Proton-gated ion channel \\ 
5mzt & Proton-gated ion channel \\ 
4hfe & Proton-gated ion channel \\ 
4hfi & Proton-gated ion channel \\ 
4ila & Proton-gated ion channel \\ 
4ilb & Proton-gated ion channel \\ 
4hfh & Proton-gated ion channel \\ 
4hfb & Proton-gated ion channel \\ 
4ilc & Proton-gated ion channel \\ 
4qh5 & Proton-gated ion channel \\ 
4p02 & Cellulose synthase catalytic subunit [UDP-forming] \\ 
4p00 & Cellulose synthase catalytic subunit [UDP-forming] \\ 
4hfd & Proton-gated ion channel \\ 
5eiy & Cellulose synthase catalytic subunit [UDP-forming] \\ 
4zzc & Proton-gated ion channel \\ 
4il9 & Proton-gated ion channel \\ 
6yj4 & NADH-ubiquinone oxidoreductase chain 3 \\ 
3rw0 & Ion\_trans domain-containing protein \\ 
4zxh & Carrier domain-containing protein \\ 
5klv & Cytochrome b-c1 complex subunit 1, mitochondrial \\ 
5okd & Cytochrome b-c1 complex subunit 1, mitochondrial \\ 
    \end{tabular}
    \caption{PDBIDs and molecules names used in the analysis of PC}
    \label{tab:my_label}
\end{table}

\begin{table}[]
    \centering
    \begin{tabular}{c|c}
PDBID & molecule name \\
\hline
5vb2 & Ion\_trans domain-containing protein \\ 
4ms2 & Ion\_trans domain-containing protein \\ 
5vb8 & Ion\_trans domain-containing protein \\ 
6mwa & Ion\_trans domain-containing protein \\ 
6mwg & Ion\_trans domain-containing protein \\ 
6mwd & Ion\_trans domain-containing protein \\ 
6mvv & Ion\_trans domain-containing protein \\ 
6haw & Cytochrome b-c1 complex subunit 1, mitochondrial \\ 
6mwb & Ion\_trans domain-containing protein \\ 
6mvw & Ion\_trans domain-containing protein \\ 
6c1k & Ion\_trans domain-containing protein \\ 
5yua & Ion\_trans domain-containing protein \\ 
6c1p & Ion\_trans domain-containing protein \\ 
6c1e & Ion\_trans domain-containing protein \\ 
6c1m & Ion\_trans domain-containing protein \\ 
6p6x & Ion\_trans domain-containing protein \\ 
6p6y & Ion\_trans domain-containing protein \\ 
6ubs & Glycine receptor subunit alphaZ1 \\ 
6mvy & Ion\_trans domain-containing protein \\ 
6xvf & Cytochrome b-c1 complex subunit 1, mitochondrial \\ 
2h26 & T-cell surface glycoprotein CD1b \\ 
3b7q & CRAL-TRIO domain-containing protein YKL091C \\ 
3b7z & CRAL-TRIO domain-containing protein YKL091C \\ 
6z5r & Light-harvesting complex 1 alpha chain \\ 
6z5s & Light harvesting complex 1 Protein W \\ 
6c15 & T-cell surface glycoprotein CD1c \\ 
1dmh & Catechol 1,2-dioxygenase \\ 
1dlm & Catechol 1,2-dioxygenase \\ 
1dlt & Catechol 1,2-dioxygenase \\ 
1dlq & Catechol 1,2-dioxygenase \\ 
3n9t & INTRADIOL\_DIOXYGENAS domain-containing protein \\ 
5kym & 1-acyl-sn-glycerol-3-phosphate acyltransferase \\ 
2hh1 & Reaction center protein L chain \\ 
2hg9 & Reaction center protein L chain \\ 
5mzq & Proton-gated ion channel \\ 
6qpc & Anoctamin-6 \\ 
6qp6 & Anoctamin-6 \\ 
5irz & Transient receptor potential cation channel subfamily V member 1 \\ 
5irx & Transient receptor potential cation channel subfamily V member 1 \\ 
    \end{tabular}
    \caption{PDBIDs and molecules names used in the analysis of PC}
    \label{tab:my_label}
\end{table}

\begin{table}[]
    \centering
    \begin{tabular}{c|c}
PDBID & molecule name \\
\hline
4i9j & 1-phosphatidylinositol phosphodiesterase \\ 
5ej1 & Cellulose synthase catalytic subunit [UDP-forming] \\ 
6s7o & Dolichyl-diphosphooligosaccharide--protein glycosyltransferase subunit STT3A \\ 
6c26 & Dolichyl-diphosphooligosaccharide--protein glycosyltransferase subunit STT3 \\ 
6s7t & Dolichyl-diphosphooligosaccharide--protein glycosyltransferase subunit STT3B \\ 
1lsh & Vitellogenin \\ 
6uz8 & Short transient receptor potential channel 6 \\ 
4tso & DELTA-actitoxin-Afr1a \\ 
4tsp & DELTA-actitoxin-Afr1a \\ 
4tsq & DELTA-actitoxin-Afr1a \\ 
6iej & Cytosolic phospholipase A2 \\ 
2a1l & Phosphatidylinositol transfer protein beta isoform \\ 
1t27 & Phosphatidylinositol transfer protein alpha isoform \\ 
4f2a & Cholesteryl ester transfer protein \\ 
4nab & Sarcoplasmic/endoplasmic reticulum calcium ATPase 1 \\ 
4uu1 & Sarcoplasmic/endoplasmic reticulum calcium ATPase 1 \\ 
5d3i & Toll-like receptor 2 \\ 
5uph & Lysosome membrane protein 2 \\ 
5ylu & Potassium-transporting ATPase alpha chain 1 \\ 
2obd & Cholesteryl ester transfer protein \\ 
5xa7 & Sarcoplasmic/endoplasmic reticulum calcium ATPase 1 \\ 
5xab & Sarcoplasmic/endoplasmic reticulum calcium ATPase 1 \\ 
5xaa & Sarcoplasmic/endoplasmic reticulum calcium ATPase 1 \\ 
6yu4 & Sodium-dependent transporter \\ 
6vyk & Mechanosensitive channel MscS \\ 
6rld & Small-conductance mechanosensitive channel \\ 
6jxi & Potassium-transporting ATPase alpha chain 1 \\ 
6jxh & Potassium-transporting ATPase alpha chain 1 \\ 
6jxj & Potassium-transporting ATPase alpha chain 1 \\ 
5xa8 & Sarcoplasmic/endoplasmic reticulum calcium ATPase 1 \\ 
5xa9 & Sarcoplasmic/endoplasmic reticulum calcium ATPase 1 \\ 
5ncq & Sarcoplasmic/endoplasmic reticulum calcium ATPase 1 \\ 
5ylv & Potassium-transporting ATPase alpha chain 1 \\ 
7d91 & Sodium/potassium-transporting ATPase subunit alpha-1 \\ 
7k4b & Transient receptor potential cation channel subfamily V member 6 \\ 
7ddl & Sodium/potassium-transporting ATPase subunit alpha-1 \\ 
7bt2 & Sarcoplasmic/endoplasmic reticulum calcium ATPase 2 \\ 
7ddh & Sodium/potassium-transporting ATPase subunit alpha-1 \\ 
7ddj & Sodium/potassium-transporting ATPase subunit alpha-1 \\ 
7ddk & Sodium/potassium-transporting ATPase subunit alpha-1 \\ 
7k4a & Transient receptor potential cation channel subfamily V member 6 \\ 
7ado & ER membrane protein complex subunit 1 \\ 
7d94 & Sodium/potassium-transporting ATPase subunit alpha-1 \\ 
6pw5 & Ion\_trans domain-containing protein \\ 
    \end{tabular}
    \caption{PDBIDs and molecules names used in the analysis of PC}
    \label{tab:my_label}
\end{table}

\begin{table}[]
    \centering
    \begin{tabular}{c|c}
PDBID & molecule name \\
\hline
3cxh & Cytochrome b-c1 complex subunit 1, mitochondrial \\
3cx5 & Cytochrome b-c1 complex subunit 1, mitochondrial \\ 
5klv & Cytochrome b-c1 complex subunit 1, mitochondrial \\ 
4rpe & Linoleate 9/13-lipoxygenase \\ 
6ymx & Cytochrome c oxidase subunit 1 \\ 
6giq & Cytochrome b-c1 complex subunit 1, mitochondrial \\ 
6nhg & Cytochrome b-c1 complex subunit 1, mitochondrial \\ 
3ar5 & Sarcoplasmic/endoplasmic reticulum calcium ATPase 1 \\ 
3ar4 & Sarcoplasmic/endoplasmic reticulum calcium ATPase 1 \\ 
2dqs & Sarcoplasmic/endoplasmic reticulum calcium ATPase 1 \\ 
2agv & Sarcoplasmic/endoplasmic reticulum calcium ATPase 1 \\ 
3jtc & Endothelial protein C receptor \\ 
3nan & Calcium-transporting ATPase \\ 
3w5c & Sarcoplasmic/endoplasmic reticulum calcium ATPase 1 \\ 
3nam & Calcium-transporting ATPase \\ 
3w5b & Sarcoplasmic/endoplasmic reticulum calcium ATPase 1 \\ 
3zuy & Transporter \\ 
3zux & Transporter \\ 
3w5a & Sarcoplasmic/endoplasmic reticulum calcium ATPase 1 \\ 
2eau & Sarcoplasmic/endoplasmic reticulum calcium ATPase 1 \\ 
3ar3 & Sarcoplasmic/endoplasmic reticulum calcium ATPase 1 \\ 
3ar7 & Sarcoplasmic/endoplasmic reticulum calcium ATPase 1 \\ 
3ar6 & Sarcoplasmic/endoplasmic reticulum calcium ATPase 1 \\ 
1l8j & Endothelial protein C receptor \\ 
1lqv & Endothelial protein C receptor \\ 
4cz8 & Na\_H\_Exchanger domain-containing protein \\ 
4ycn & Sarcoplasmic/endoplasmic reticulum calcium ATPase 1 \\ 
5jmn & Multidrug efflux pump subunit AcrB \\ 
4j2t & Sarcoplasmic/endoplasmic reticulum calcium ATPase 1 \\ 
3w5d & Sarcoplasmic/endoplasmic reticulum calcium ATPase 1 \\ 
4v3e & IT4VAR07 CIDRA \\ 
4v3d & HB3VAR03 CIDRA DOMAIN \\ 
5a1s & Citrate-sodium symporter \\ 
4ycm & Sarcoplasmic/endoplasmic reticulum calcium ATPase 1 \\ 
4n31 & Signal peptidase I \\ 
2vwa & Early transcribed membrane protein 13 \\ 
3pg7 & Neurofibromin \\ 
5td3 & Catechol 1,2-dioxygenase \\ 
5vxt & Catechol 1,2-dioxygenase \\ 
6wlw & V-type proton ATPase 21 kDa proteolipid subunit \\ 
6wbn & Pannexin-1 \\ 
6wbg & Pannexin-1 \\ 
6wbf & Pannexin-1 \\ 
6wbm & Pannexin-1 \\ 
    \end{tabular}
    \caption{PDBIDs and molecules names used in the analysis of PC}
    \label{tab:my_label}
\end{table}

\begin{table}[]
    \centering
    \begin{tabular}{c|c}
PDBID & name \\
\hline
6z16 & Multisubunit Na+/H+ antiporter, A subunit \\ 
6ymy & Cytochrome c oxidase subunit 1 \\ 
6wm2 & V-type proton ATPase subunit E 1 \\ 
1p84 & Ubiquinol-cytochrome C reductase complex core protein I \\ 
3szd & Probable porin \\ 
3m9i & Lens fiber major intrinsic protein \\ 
5ejz & Cellulose synthase catalytic subunit [UDP-forming] \\ 
4p02 & Cellulose synthase catalytic subunit [UDP-forming] \\ 
4pd4 & Cytochrome b-c1 complex subunit 1, mitochondrial \\ 
4p00 & Cellulose synthase catalytic subunit [UDP-forming] \\ 
4qn9 & N-acyl-phosphatidylethanolamine-hydrolyzing phospholipase D \\ 
5ej1 & Cellulose synthase catalytic subunit [UDP-forming] \\ 
6zk9 & NADH dehydrogenase [ubiquinone] flavoprotein 1, mitochondrial \\ 
6zkb & Mitochondrial complex I, 49 kDa subunit \\ 
1m57 & Cytochrome c oxidase subunit 1 \\ 
1n69 & Saposin-B \\ 
1m56 & Cytochrome c oxidase subunit 1 \\ 
5weh & Cytochrome c oxidase subunit 1 \\ 
6zqv & Genome polyprotein \\ 
2x72 & Rhodopsin \\ 
2qgu & Probable signal peptide protein \\ 
3wmm & Photosynthetic reaction center cytochrome c subunit \\ 
1yp0 & Steroidogenic factor 1 \\ 
1zdt & Steroidogenic factor 1 \\ 
1kb9 & Cytochrome b-c1 complex subunit 1, mitochondrial \\ 
1eys & PHOTOSYNTHETIC REACTION CENTER \\ 
5y5s & Photosynthetic reaction center cytochrome c subunit \\ 
5w7l & Bac\_transf domain-containing protein \\ 
6t15 & Cytochrome b-c1 complex subunit 1, mitochondrial \\ 
5z39 & PH domain-containing protein \\ 
6hu9 & Cytochrome b-c1 complex subunit 1, mitochondrial \\ 
5irz & Transient receptor potential cation channel subfamily V member 1 \\ 
5irx & Transient receptor potential cation channel subfamily V member 1 \\ 
6cud & Short transient receptor potential channel 3 \\ 
6o72 & Transient receptor potential cation channel subfamily M member 8 \\ 
6o6r & Transient receptor potential cation channel subfamily M member 8 \\ 
2e2x & Neurofibromin \\ 
3peg & Neurofibromin \\ 
3p7z & Neurofibromin \\ 
    \end{tabular}
    \caption{PDBIDs and molecules names used in the analysis of PC}
    \label{tab:my_label}
\end{table}

\begin{table}[]
    \centering
    \begin{tabular}{c|c}
PDBID & name \\
\hline
6rd5 & ASA-10: Polytomella F-ATP synthase associated subunit 10 \\ 
6xiw & Sodium leak channel non-selective protein \\ 
6lum & Succinate dehydrogenase subunit C \\ 
6tyi & Biopolymer transport protein ExbB \\ 
5jwy & Phosphatidylglycerophosphatase B \\ 
6uza & Short transient receptor potential channel 6 \\ 
3l9r & Ig-like domain-containing protein \\ 
6or2 & Membrane protein, MmpL family protein \\ 
6w2y & Gamma-aminobutyric acid type B receptor subunit 1 \\ 
2ein & Cytochrome c oxidase subunit 1 \\ 
2eil & Cytochrome c oxidase subunit 1 \\ 
2eij & Cytochrome c oxidase subunit 1 \\ 
3ag3 & Cytochrome c oxidase subunit 1 \\ 
3abm & Cytochrome c oxidase subunit 1 \\ 
2y69 & Cytochrome c oxidase subunit 1 \\ 
2eik & Cytochrome c oxidase subunit 1 \\ 
2dyr & Cytochrome c oxidase subunit 1 \\ 
2r40 & Ecdysone receptor \\ 
1r20 & NR LBD domain-containing protein \\ 
3ae5 & Succinate dehydrogenase [ubiquinone] flavoprotein subunit, mitochondrial \\ 
3ixp & NR LBD domain-containing protein \\ 
3vr8 & Succinate dehydrogenase [ubiquinone] flavoprotein subunit, mitochondrial \\ 
3vr9 & Succinate dehydrogenase [ubiquinone] flavoprotein subunit, mitochondrial \\ 
1zoy & Succinate dehydrogenase [ubiquinone] flavoprotein subunit, mitochondrial \\ 
1g2n & NR LBD domain-containing protein \\ 
1nek & Succinate dehydrogenase flavoprotein subunit \\ 
1r1k & Ecdysone receptor \\ 
3ae8 & Succinate dehydrogenase [ubiquinone] flavoprotein subunit, mitochondrial \\ 
3ae9 & Succinate dehydrogenase [ubiquinone] flavoprotein subunit, mitochondrial \\ 
3aeb & Succinate dehydrogenase [ubiquinone] flavoprotein subunit, mitochondrial \\ 
3ae2 & Succinate dehydrogenase [ubiquinone] flavoprotein subunit, mitochondrial \\ 
3ae4 & Succinate dehydrogenase [ubiquinone] flavoprotein subunit, mitochondrial \\ 
3ae3 & Succinate dehydrogenase [ubiquinone] flavoprotein subunit, mitochondrial \\ 
3abv & Succinate dehydrogenase [ubiquinone] flavoprotein subunit, mitochondrial \\ 
3ae6 & Succinate dehydrogenase [ubiquinone] flavoprotein subunit, mitochondrial \\ 
3aea & Succinate dehydrogenase [ubiquinone] flavoprotein subunit, mitochondrial \\ 
3ae1 & Succinate dehydrogenase [ubiquinone] flavoprotein subunit, mitochondrial \\ 
3aef & Succinate dehydrogenase [ubiquinone] flavoprotein subunit, mitochondrial \\ 
1nen & Succinate dehydrogenase flavoprotein subunit \\ 
4ytm & Succinate dehydrogenase flavoprotein \\ 
3vra & Succinate dehydrogenase [ubiquinone] flavoprotein subunit, mitochondrial \\ 
3vrb & Succinate dehydrogenase [ubiquinone] flavoprotein subunit, mitochondrial \\ 
    \end{tabular}
    \caption{PDBIDs and molecules names used in the analysis of PC}
    \label{tab:my_label}
\end{table}

\begin{table}[]
    \centering
    \begin{tabular}{c|c}
PDBID & name \\
\hline
4g33 & Linoleate 9/13-lipoxygenase \\ 
5ir4 & Lipoxygenase LoxA \\ 
5lc8 & Linoleate 9/13-lipoxygenase \\ 
5ir5 & Lipoxygenase LoxA \\ 
4g32 & Linoleate 9/13-lipoxygenase \\ 
6a9j & Autophagy-related protein 2 \\ 
6adq & Cytochrome aa3 subunit 2 \\ 
7bh1 & Potassium-transporting ATPase potassium-binding subunit \\ 
7bgy & Potassium-transporting ATPase potassium-binding subunit \\ 
7lc3 & Potassium-transporting ATPase potassium-binding subunit \\ 
7bh2 & Potassium-transporting ATPase potassium-binding subunit \\ 
5jxd & Tumor necrosis factor alpha-induced protein 8 \\ 
5h5a & Mitochondrial distribution and morphology protein 12 \\ 
6pqo & Transient receptor potential cation channel subfamily A member 1 \\ 
6pqp & Transient receptor potential cation channel subfamily A member 1 \\ 
6x16 & Glutamate transporter homologue GltPh \\ 
6vyi & Diacylglycerol O-acyltransferase 1 \\ 
6w98 & F5/8 type C domain-containing protein \\ 
6uz3 & Sodium channel protein type 5 subunit alpha,Green fluorescent protein \\ 
6uw4 & Transient receptor potential cation channel subfamily V member 3 \\ 
6x15 & Glutamate transporter homologue GltPh \\ 
6agf & Sodium channel protein type 4 subunit alpha \\ 
6br8 & Protein A6 homolog \\ 
6br9 & Protein A6 homolog \\ 
6vz1 & Diacylglycerol O-acyltransferase 1 \\ 
6uwf & Glutamate transporter homolog \\ 
6uz0 & Sodium channel protein type 5 subunit alpha,Green fluorescent protein \\ 
6pqq & Transient receptor potential cation channel subfamily A member 1 \\ 
7k18 & Green fluorescent protein \\ 
7kv8 & Envelope protein E \\ 
7cu3 & Sodium leak channel non-selective protein \\ 
2qjk & Cytochrome b \\ 
2qjy & Cytochrome b \\ 
2fyn & Cytochrome b \\ 
3ayg & COX1 domain-containing protein \\ 
2qjp & Cytochrome b \\ 
3ayf & COX1 domain-containing protein \\ 
5kkz & Cytochrome b \\ 
5kli & Cytochrome b \\ 
6qq6 & Nitric oxide reductase subunit B \\ 
1pp9 & Ubiquinol-cytochrome C reductase complex core protein I, mitochondrial \\ 
    \end{tabular}
    \caption{PDBIDs and molecules names used in the analysis of PC}
    \label{tab:my_label}
\end{table}

\begin{table}[]
    \centering
    \begin{tabular}{c|c}
PDBID & name \\
\hline
3h1l & Mitochondrial ubiquinol-cytochrome-c reductase complex core protein i \\ 
3tgu & Mitochondrial ubiquinol-cytochrome-c reductase complex core protein i \\ 
3h1k & Mitochondrial ubiquinol-cytochrome-c reductase complex core protein i \\ 
3l70 & Mitochondrial ubiquinol-cytochrome-c reductase complex core protein i \\ 
3l75 & Mitochondrial ubiquinol-cytochrome-c reductase complex core protein i \\ 
3l71 & Mitochondrial ubiquinol-cytochrome-c reductase complex core protein i \\ 
2bcc & Cytochrome b-c1 complex subunit 1, mitochondrial \\  
    \end{tabular}
    \caption{PDBIDs and molecules names used in the analysis of PC}
    \label{tab:my_label}
\end{table}

\begin{table}[]
    \centering
    \begin{tabular}{c|c}
PDBID & molecule name \\
\hline
3mtx & Lymphocyte antigen 86 \\ 
4c7r & Glycine betaine transporter BetP \\ 
4doj & Glycine betaine transporter BetP \\ 
5azb & Phosphatidylglycerol--prolipoprotein diacylglyceryl transferase \\ 
4opc & Digeranylgeranylglycerophospholipid reductase \\ 
4phz & unknown peptide \\ 
5azc & Phosphatidylglycerol--prolipoprotein diacylglyceryl transferase \\ 
6z5r & Light-harvesting complex 1 alpha chain \\ 
6z5s & Light harvesting complex 1 Protein W \\ 
6tyi & Biopolymer transport protein ExbB \\ 
6tjv & NAD(P)H-quinone oxidoreductase subunit 1 \\ 
2hhk & Reaction center protein L chain \\ 
2cyd & V-type sodium ATPase subunit K \\ 
2bhw & Chlorophyll a-b binding protein AB80, chloroplastic \\ 
1rwt & Chlorophyll a-b binding protein, chloroplastic \\ 
2axt & Photosystem II protein D1 1 \\ 
2bl2 & V-type sodium ATPase subunit K \\ 
3jcu & Photosystem II protein D1 \\ 
5t85 & Antibody 10E8 FAB HEAVY CHAIN \\ 
2ein & Cytochrome c oxidase subunit 1 \\ 
2eil & Cytochrome c oxidase subunit 1 \\ 
2eij & Cytochrome c oxidase subunit 1 \\ 
2irv & Rhomboid protease GlpG \\ 
3ag3 & Cytochrome c oxidase subunit 1 \\ 
3abm & Cytochrome c oxidase subunit 1 \\ 
2y69 & Cytochrome c oxidase subunit 1 \\ 
2eik & Cytochrome c oxidase subunit 1 \\ 
2dyr & Cytochrome c oxidase subunit 1 \\ 
3oz2 & Digeranylgeranylglycerophospholipid reductase \\ 
4i7z & Cytochrome b6 \\ 
5wki & T-cell surface glycoprotein CD1b \\ 
5wl1 & T-cell surface glycoprotein CD1b \\ 
2r9r & Voltage-gated potassium channel subunit beta-2 \\ 
3wmm & Photosynthetic reaction center cytochrome c subunit \\ 
4jta & Voltage-gated potassium channel subunit beta-2 \\ 
4jtc & Voltage-gated potassium channel subunit beta-2 \\ 
4jtd & Voltage-gated potassium channel subunit beta-2 \\ 
3lnm & Voltage-gated potassium channel subunit beta-2 \\ 
\end{tabular}
    \caption{PDBIDs and molecule names of PG lipids}
    \label{tab:my_label}
\end{table}


\begin{table}[]
    \centering
    \begin{tabular}{c|c}
PDBID & name \\
\hline
6tos & Orexin receptor type 1 \\ 
6to7 & Orexin receptor type 1 \\ 
6tot & Orexin receptor type 1 \\ 
6tp6 & Orexin receptor type 1 \\ 
6tod & Orexin receptor type 1 \\ 
6cju & Cyclic nucleotide-binding domain-containing protein \\ 
6cjt & Cyclic nucleotide-binding domain-containing protein \\ 
6bw5 & UDP-N-acetylglucosamine--dolichyl-phosphate N-acetylglucosaminephosphotransferase \\ 
6bw6 & UDP-N-acetylglucosamine--dolichyl-phosphate N-acetylglucosaminephosphotransferase \\ 
6tp4 & Orexin receptor type 1 \\ 
6tq6 & Orexin receptor type 1 \\ 
6tq7 & Orexin receptor type 1 \\ 
6tq4 & Orexin receptor type 1 \\ 
6v35 & Calcium-activated potassium channel subunit alpha-1 \\ 
6tp3 & Orexin receptor type 1 \\ 
6tq9 & Orexin receptor type 1 \\ 
7cge & Lipid asymmetry maintenance ABC transporter permease subunit MlaE \\ 
1ymt & Steroidogenic factor 1 \\ 
7kc4 & Protein Wnt-8a \\ 
1yok & Nuclear receptor subfamily 5 group A member 2 \\ 
4nh2 & Ammonium transporter \\ 
4oni & Nuclear receptor subfamily 5 group A member 2 \\ 
5eg1 & Microcin-J25 export ATP-binding/permease protein McjD \\ 
3tx7 & Catenin beta-1 \\ 
6rck & Outer membrane protein C \\ 
6uxp & Bcl-2 homologous antagonist/killer \\ 
6hsy & Toluene tolerance protein Ttg2D \\ 
\end{tabular}
    \caption{PDBIDs and molecule names of PG lipids}
    \label{tab:my_label}
\end{table}


\begin{table}[]
    \centering
    \begin{tabular}{c|c}
PDBID & name \\
\hline
3kaa & Hepatitis A virus cellular receptor 2 homolog \\ 
3bib & T-cell immunoglobulin and mucin domain-containing protein 4 \\ 
1dsy & Protein kinase C alpha type \\ 
6wlw & V-type proton ATPase 21 kDa proteolipid subunit \\ 
6wm2 & V-type proton ATPase subunit E 1 \\ 
6b8o & Triggering receptor expressed on myeloid cells 2 \\ 
6snd & LN01 light chain \\ 
6roh & Probable phospholipid-transporting ATPase DRS2 \\ 
3d9s & Aquaporin-5 \\ 
5dye & Aquaporin-5 \\ 
5c5x & Aquaporin-5 \\ 
6u9w & P2X purinoceptor 7 \\ 
6u9v & P2X purinoceptor 7 \\ 
4b2z & Oxysterol-binding protein homolog 6 \\ 
6lcp & Phospholipid-transporting ATPase \\ 
6lcr & Phospholipid-transporting ATPase \\ 
6i3y & PRELI domain-containing protein 1, mitochondrial \\ 
6vp0 & Diacylglycerol O-acyltransferase 1 \\ 
6tt7 & ATP synthase subunit alpha \\ 
6sp2 & Membrane protein TMS1d \\ 
7bsu & ATP11C \\ 
7bsv & ATP11C \\ 
6k7m & Phospholipid-transporting ATPase \\ 
2hj6 & Reaction center protein L chain \\ 
4hyt & Sodium/potassium-transporting ATPase subunit alpha-1 \\ 
4res & Sodium/potassium-transporting ATPase subunit alpha-1 \\ 
6uxn & Bcl-2 homologous antagonist/killer \\ 
6nhh & Cytochrome b \\ 
    \end{tabular}
    \caption{PDBIDs and molecule names of protein structures used in the analysis of PS headgroup structures.}
    \label{tab:my_label}
\end{table}

\end{document}
